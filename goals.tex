
\section{Ziel der Arbeit}
\label{sec:ziel_der_arbeit}

Im Modul \textit{relax-concentrate-create} (rcc) erfassen Studierende ihre Tätigkeiten in Bezug auf
regenerative, konzentrative und kreative Ressourcen. Diese Daten sollen nun automatisiert analysiert
und ausgewertet werden um den Studierenden bessere Hinweise zu ihrem Ressourcenmanagement geben zu
können. 

Das Ziel der Arbeit ist erstens die Analyse von möglichen Auswertungen und statistischen Methoden,
die auf der Datenbasis angewendet werden können. In einem zweiten Schritt geht es darum, einen
Prototypen zu erstellen, der eine Basis bildet, auf welcher diese Auswertungen erstellt werden
können. Zudem sollen einige der Auswertungen bereits gemacht werden können.

\subsection{Problemstellung}
\label{sub:problemstellung}

Die Auswertung der Daten geschieht aktuell weitgehend manuell. In dieser Arbeit soll versucht
werden, Teile der Auswertungen zu automatisieren. Einerseits geht es darum, bereits bestehende Ideen
zu konkretisieren. Andererseits soll auch untersucht werden, ob mit weiteren Methoden sinnvolle
Resultate erbracht werden können.



\subsection{Erwartete Resultate}
\label{sub:erwartete_resultate}

